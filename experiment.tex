\section{\label{sec:experiment}Experiments}
%\gn{You need at least one sentence of lead-in here.}
In this section, we first discuss the experimental results of DDS on image classification, an instance of the general classification problem using Algorithm \ref{alg:image_classification_dds}. Then we demonstrate the effectiveness of DDS on a special case of multilingual NMT using Algorithm \ref{alg:nmt_dds}.

\subsection{\label{exp:image_classification}Image Classification}

\paragraph{Settings.} We apply our method on two established image classification datasets, CIFAR-10~\citep{cifar10} and ImageNet~\citep{imagenet}. For each dataset, we consider two settings: a reduced setting where only roughly 10\% of the training labels are used, and a full setting, where all labels are used. Specifically, the reduced setting for CIFAR-10 uses the first $4000$ examples in the training set, and with ImageNet, the reduced setting uses the first $102$ TFRecord shards as pre-processed by~\citet{imagenet_generalize_better}. We use the size of $224 \times 224$ for ImageNet.

\paragraph{Models and Baselines.} For CIFAR-10, we use the pre-activation WideResNet-28~\citep{wide_res_net}, with a width factor $k=2$ for the reduced setting and $k=10$ for the normal setting. For ImageNet, we use the post-activation ResNet-50~\citep{res_net}. Our baselines are the standard implementations of these models, where we reproduce the numbers commonly reported in the literature~\citep{wide_res_net,res_net,resnext}.

\paragraph{Training details.} The batch sizes for CIFAR-10 and for ImageNet are $128$ and $4096$, running for 200K steps and 40K steps, respectively. We use the standard Momentum update for the model parameter $\theta$, and the derived Momentum update rule in Section~\ref{sec:grad_of_optimizers} for the distribution parameter $\psi$ \gn{I feel that ``distribution parameter'' and ``distribution model'' are newly introduced in this section? Maybe ``DDS parameter'' and ``DDS model'' would be easier to understand?}, both with the momentum rate of $0.9$. Additional training details can be found in Appendix \ref{app:image_hparam}.

Our experiments are conducted on the second generation of Tensor Processing Units (TPUv2). We discuss three important implementation details to improve the training efficiency of ImageNet models. First, each batch of $4096$ training instances for ImageNet is processed in parallel on $32$ TPU cores, each working on $128$ images. When we compute $p(\hat{x}, \hat{y}; \psi)$ (in~Section \ref{sec:image_method} \gn{This used to be ``\cf'', but ``\cf'' means ``in contrast to'', so this is not proper usage.}), the softmax function is computed \textit{locally on each core} to reduce the synchronization overhead. Second, since we do not need the parameters $\psi$ of the distribution model, we ignore all batch normalization moving average updates when we pass images through $p(\hat{x}, \hat{y}; \psi)$. We also only batch-normalize the distribution model locally on each TPU core. Controlled profiling measures show that the aforementioned details speed up the training process by almost $2.5 \times$. Third, following~\citet{neural_combi} and~\citet{enas}, for ImageNet, we apply a $\tanh$ activation to the logits prior to the softmax to compute $p(\hat{x}, \hat{y}; \psi)$, which softens the softmax distribution and prevents the $p(\hat{x}, \hat{y}; \psi)$ from collapsing into always choosing a particular example \gn{Just a comment: I'm not a huge fan of this because it seems a bit hacky, but if it's what was necessary to make things work this time then that's fair}. We note that all these three tricks are not needed for our experiments on CIFAR-10, where the batch size is much smaller.

\begin{wraptable}{r}{0.6\textwidth}
%\begin{adjustbox}{max width=0.7\textwidth}
\vspace{-1cm}
\resizebox{0.6\textwidth}{!}{
  \begin{tabular}{llll}
    \multicolumn{4}{c}{\textbf{CIFAR-10} ($\text{mean} \pm \text{std}$ over $10$ runs)} \\
  \toprule
    \textbf{Portion} &
    \textbf{Model} &
    \textbf{Baseline} &
    \textbf{\dds}
    \\
  \midrule
    4K &
    WideResNet-28-2 &
    $82.60 \pm 0.17$ & % 6109331
    $\mathbf{83.63 \pm 0.29}$ % 6109486
    \\
    Full &
    WideResNet-28-10 &
    $95.55 \pm 0.15$ &  % 6153155
    $\mathbf{96.31 \pm 0.13}$  % 6170380
    \\
  \bottomrule
    \\
    \multicolumn{4}{c}{\textbf{ImageNet} ($\text{Top-1}/\text{Top-5}$)} \\
    \toprule
    \textbf{Portion} &
    \textbf{Model} &
    \textbf{Baseline} &
    \textbf{\dds}
    \\
  \midrule
    10\%  & ResNet-50 &
    $56.36 / 79.45$ & % 5267621/workUnits/118
    $\mathbf{56.81 / 79.51}$   % 6020481/workUnits/18
    \\  
    Full & ResNet-50 &
    $76.51 / 93.20$ & % https://arxiv.org/abs/1810.12890
    $\mathbf{77.23 / 93.57}$ % 6800564/workUnits/39
    \\
    \bottomrule
  \end{tabular}
}
  \captionof{table}{\label{tab:image_classification_results}Image classification accuracy. Higher is better.}
%\end{adjustbox}
%\end{center}
\end{wraptable}

\paragraph{Results.} Results are presented in Table~\ref{tab:image_classification_results}. As can be seen, \dds~improves the performance of all tasks considered. \gn{The following explanation is pretty weak because increasing the degrees of freedom of the model was not even one of our original motivations. Is there any way we can at least say something about how it (might be?) reducing the weight on outliers or something like that?} In our intuitions, \dds~provides an extra degree of freedom to train the models, namely the per-example scaling of gradients. This extra degree of freedom is well-utilized by the $p(\hat{x}, \hat{y}; \psi)$ distribution, leading to the improvements.


\subsection{Multilingual NMT}
\paragraph{Dataset.}
We use the 58-language-to-English TED dataset~\citep{ted_pretrain_emb}. 
Following setups in prior work~\citep{ted_pretrain_emb,rapid_adapt_nmt,SDE}, we use three low-resource languages~(LRL) Azerbaijani~(aze), Belarusian~(bel), Galician~(glg) to English,
and a slightly higher-resource dataset, Slovak~(slk) to English. Each language is paired with its most related high-resource language~(HRL), namely Turkish~(tur), Russian~(rus), Portugese~(por), and Czech~(ces) respectively, with the statistics of the datasets listed in Appendix~\ref{app:nmt_data}. Our goal is to optimize the usage of the data from the 8 languages for training NMT models for each LRL. 

\paragraph{Baselines.} We compare our method against three strong baselines: 1) All: all 8 languages are used for training without any data selection; 2) Bi: we train on the combined datasets of each LRL and its related HRL, which is a special case of data selection that requires prior linguistic knowledge; 3) TCS~\citep{TCS}: the state-of-the-art data selection method for multilingual NMT. Given a target sentence, TCS conditionally samples a source sentence from the candidate languages based on simple heuristics such as vocabulary overlap.

\paragraph{Implementation.} Here we clarify several design choices for Algorithm \ref{alg:nmt_dds}. To model the distribution $p(S_i|y;\psi)$, we use a 2-layer feed-forward network with hidden size of 32. The input vector for the network is a vector of size of source languages $n$, representing which of the languages contain a corresponding source sentence for a given target sentence $y$. We use the standard Adam update rule with learning rate of 0.001 for the NMT model parameter $\theta$, and the derived Adam update rule from Section~\ref{sec:grad_of_optimizers} with learning rate~0.0001 for the distribution parameter $\psi$. 

We test two different settings for using DDS for multilingual  NMT: 1) TCS+DDS: we pretrain the network with the heuristic distribution from TCS before the DDS training process
%. K in Algorithm \ref{alg:nmt_dds} is set to be 50,000; 
2) Uniform+DDS: we train the network for $\psi$ from scratch. %At the start of training, K is set to be 5,000 to encourage exploration; after updating $\psi$ for 5 times, we also set it to 50,000.
We run all experiments with 3 different random seeds and pick the median for each method. Additional hyperparameters are listed in Appendix~\ref{app:nmt_hparam}.

\paragraph{Results.}
\begin{wraptable}{l}{0.6\textwidth}
\begin{center}
\vspace{-0.5cm}
    \begin{tabular}{l|cccc}
    \toprule
    \textbf{Model} & \textbf{aze} & \textbf{bel} & \textbf{glg} & \textbf{slk} \\
    \midrule
    All & 10.31 & 17.21 & 26.05 & 27.44 \\
    Bi & 10.34 & 15.31 & 27.41 & 25.92 \\
    TCS & 11.18 & 16.97 & 27.28 & 27.72 \\
    \midrule
    TCS+DDS & \textbf{11.84} & \textbf{17.74} & \textbf{27.78} & 27.74 \\
    %Uniform & 9.54 & 14.75 & 25.11 & 26.30 \\
    Uniform+DDS & 10.74 & 17.24 & 27.32 & \textbf{28.20} \\
    \bottomrule
    \end{tabular}
     \captionof{table}{\label{tab:nmt_result}BLEU score. Higher is better.}
\end{center}
\vspace{-0.5cm}
\end{wraptable}
The results of the baselines and our method are listed in Table \ref{tab:nmt_result}. In our setting, training on all 8 languages (All) performs better than using a single related language (Bi) on~2 of the~4 languages, probably because the number of languages in the training corpus is not very large. However, the All method requires significantly more training steps than the other methods, and has found to be worse than Bi when the number of training languages is large~\citep{rapid_adapt_nmt,TCS}. The heuristic-based data selection method (TCS) effectively improves over or performs comparably with the Bi baseline. In general, DDS outperforms all strong baselines on all 4 languages. When the data selection distribution is initialized with the heuristics from TCS, it consistently improves over TCS on all 4 languages~(TCS+DDS). Even without heuristic initialization, TCS+Uniform still achieves competivitve performance compared to the baselines. Notably, for slk, the best performing method is discovered by DDS without heuristic initialization.
