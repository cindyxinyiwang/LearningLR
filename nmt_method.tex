\subsection{\label{sec:nmt_method}Formulation for Multilingual NMT}

In this section, we demonstrate an application of DDS to multilingual models for NMT, specifically applied to improving accuracy on low-resource languages (LRL)~\citep{nmt_transfer,rapid_adapt_nmt,johnson_nmt}.
In this setting, we assume that we have a particular LRL $S$ that we would like to translate into target language $T$, and we additionally have a multilingual corpus $\mathcal{D}_{\text{train}}$ that has parallel data between $n$ source languages $(S_1, S_2, ..., S_n)$ and target language $T$.
We would like to pick parallel data from any of the source languages to the target language to improve translation of a particular LRL $S$, so we assume that $\mathcal{D}_{\text{dev}}$ exclusively consists of parallel data between $S$ and $T$.
As a result, DDS will attempt to select data from $\mathcal{D}_{\text{train}}$ that improve accuracy on $S$-to-$T$ translation as represented by $\mathcal{D}_{\text{dev}}$.
% We then train a parameterized distribution $p(x, y;\psi)$ with a support over $\mathcal{D}_{\text{train}}$, such that selecting training data according to this distribution leads to the optimal NMT model $\theta^*$ on the language $S$. 
%That is, $\psi$ satisfies the following condition
%\begin{equation}
%  \label{eqn:nmt_argmin}
%  \begin{aligned}
%    \arg\min_\psi
%    \sum_{x_i, y_i \in S\text{-}Y} \ell_{\text{dev}}(x_i, y_i; \theta^*)
%    ~~~\text{where}~~~
%    \theta^* = \arg\min_\theta \mathbb{E}_{x_i,y_i \sim p(X, Y;\psi)}\left[ \ell_{\text{train}}(x_i, y_i; \theta) \right]
%  \end{aligned}
%\end{equation}
%\begin{wraptable}{l}{8cm}
    %\vspace{-0.3cm}
\begin{center}
\resizebox{!}{3.5cm}{
\begin{algorithm}[H]
\SetAlgoLined
\DontPrintSemicolon
\SetKwInOut{Input}{Input}
\SetKwInOut{Output}{Output}
\SetCommentSty{itshape}
\SetKwComment{Comment}{$\triangleright$\ }{}
%\KwResult{Write here the result }

\Input{$\mathcal{D}_{\text{train}}$; K: number of data to train the NMT model before updating $\psi$; 
E: number of updates for $\psi$; 
$\alpha_1$,$\alpha_2$: discount factors for the gradient}

\Output{The converged NMT model $\theta^*$}

  Initialize $\psi_0$, $\theta_0$
  
  \Comment{Initialize the gradient of each source language}
  $grad[S_i] \leftarrow 0$ \textbf{for} \textit{i in n}
  %\For{ i in n}{
  %
  %  $grad[S_i] \leftarrow 0$
  %  
  %}
 
  \While{$\theta$ not converged}{
    %\Comment{Sample training data according to $\psi$}
    $X, Y \leftarrow \text{load\_data}(\psi, \mathcal{D}_{\text{train}}, K)$  \label{alg:load_nmt}
  
    \Comment{Train the NMT model}
    \For{ $x_i, y$ in $X, Y$}{
      $\theta_t \leftarrow \text{GradientUpdate}\left( \theta_{t-1}, \nabla_{\theta_{t-1}} \ell(x_i, y; \theta_{t-1}) \right)$
        
      $grad[S_i] \leftarrow \alpha_1 \times \text{grad}[S_i] + \alpha_2 \times \nabla_{\theta_{t-1}} \ell(x_i, y; \theta_{t-1})$
    }
  
    \Comment{Optimize $\psi$}
    \For{ iter in E}{
      
      sample $B$ data pairs from $\mathcal{D}_{\text{train}}$
      
      $d_\psi \leftarrow \frac{1}{B} \sum_{j=1}^B \sum_{i=1}^n \left[ \text{grad}[S_i]^\top \text{grad}[S] \cdot \nabla_{\psi_{t-1}} \text{log}\left( p\left( S_i|y_j;\psi_{t-1} \right) \right) \right]$
       
      $\psi_t \leftarrow \text{GradientUpdate}(\psi_{t-1}, d_{\psi_{t-1}})$ 
    }
  }
  \caption{\label{alg:nmt_dds}Training multilingual NMT with DDS.}
\end{algorithm}
}
\end{center}
    \vspace{-0.2cm}
%\end{wraptable} 

To make training more efficient and stable in this setting, we make three simple modifications of the main framework in Section \ref{sec:diff_data_selection} that take advantage of the problem structure of multilingual NMT.
First, instead of directly modeling $p(X,Y;\psi)$, we assume a uniform distribution over the target sentence $Y$, %($p(Y)$)
and only parameterize the conditional distribution of which source language sentence to pick given the target sentence: $p(X|y;\psi)$. This design follows the formulation of Target Conditioned Sampling~(TCS~\citep{TCS}), an existing state-of-the-art data selection method with a similar structure, but models the distribution $p(X|y)$ using heuristics.
Second, we only update $\psi$ after updating the NMT model for a fixed number of steps.
Third, we sample the data according to $p(X|y;\psi)$ to get a Monte Carlo estimate of the objective in Equation \ref{eqn:psi_theta_argmin}.
This can significantly reduce the training time compared to using all data.
The pseudo code of the training process is in Algorithm \ref{alg:nmt_dds}.

%Note that in Line \ref{alg:load_nmt} of Algorithm \ref{alg:nmt_dds}, we load $K$ training data according to $p(\psi)$. Since we formulate $p(\psi)$ as $p(S_i|y;\psi)$, the data loading procedure is the same as the TCS algorithm~\citep{TCS}: for each of the $K$ target sentences, we calculate a distribution over its source languages according to $p(S_i|y;\psi)$, and then sample the corresponding source sentences based on this distribution.

 
