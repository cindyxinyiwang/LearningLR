\begin{abstract}
To acquire a new skill, humans learn better and faster if a tutor, based on their current knowledge level, informs them of how much attention they should pay to particular content or practice problems. Similarly, a machine learning model could potentially be trained better with a scorer that ``adapts'' to its current learning state and estimates the importance of each training data instance. Training such an adaptive scorer efficiently is a challenging problem; in order to precisely quantify the effect of a data instance at a given time during the training, it is typically necessary to first complete the entire training process.
To efficiently optimize data usage, we propose a reinforcement learning approach called Differentiable Data Selection~(\dds). In \dds, we formulate a scorer network as a learnable function of the training data, which can be efficiently  updated along with the main model being trained. Specifically, \dds~updates the scorer with an intuitive reward signal: it should up-weigh the data that has a similar gradient with a dev set upon which we would finally like to perform well. Without significant computing overhead, \dds~delivers strong and consistent improvements over several strong baselines on two very different tasks of machine translation and image classification.%
% two modalities. Specifically, on multilingual machine translation, \dds~can dynamically identify which related languages are most helpful to improve the translation of another language, leading to consistent improvements over a strong heuristic data selection baseline. On image classification tasks with CIFAR-10 and ImageNet, and with different ranges of data, \dds~can also optimize the importance of different training instances throughout training, leading to consistent improvements over the baselines in all settings. \gn{The abstract seemed very long, so I shortened the experimental details. Please check.}
\footnote{We will make the code publicly available upon acceptance.}

\end{abstract}