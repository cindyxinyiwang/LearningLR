\section{\label{sec:related_work}Related Work}
Many machine learning approaches consider how to best present data to models. First, difficulty-based curriculum learning estimates the presentation order based on heuristic understanding of the hardness of examples~\citep{cl_bengio,SpitkovskyAJ10,baysian_curriculum,zhang2016boosting,automate_cl_GravesBMMK17,zhang2018empirical,platanios19naacl}. These methods, though effective, often generalize poorly because they require task-specific difficulty measures. On the other hand, self-paced learning~\citep{spl_kumar,spl_visual_category} defines the hardness of the data based on the loss from the model, but is still based on the assumption that the model should learn from easy examples. Our method does not make these assumptions. Closest to the learning to teach framework~\citep{learn_to_teach} but their formulation involves manual feature design and requires expensive multi-pass optimization. Instead, we formulate our reward using bi-level optimization, which has been successfully applied for a variety of other tasks~\citep{bilevel_optim,hier_optim,darts,hyper_grad,learn_reweight}.

Data selection for domain adaptation for disparate tasks has also been extensively studied~\citep{moore2010intelligent,axelrod2011domain,domain_adapt_transfer,jiang-zhai-2007-instance,foster-etal-2010-discriminative,wang-etal-2017-instance}. These methods generally design heuristics to measure domain similarity.
%\cite{domain_adapt_transfer} propose to estimate the importance weight of the classification labels in the pretraining dataset to mitigate the domain differences, while \dds~is a more general data selection framework that works for both classification and other usage cases.
Submodular optimization~\citep{submodular_mt,learn_mix_submodular} selects training data that are similar to dev set, but the criterion is often based on hand-designed features and the data usage is predefined before training. 
Besides domain adaptation, selecting also benefits training in the face of noisy or otherwise undesirable data~\citep{vyas-etal-2018-identifying,pham-etal-2018-fixing}.

Our method is also related to works on training instance weighting~\citep{importance_weight,learn_reweight,jiang-zhai-2007-instance,domain_adapt_transfer}. These methods reweigh data based on a manually computed weight vector, instead of using a parameterized neural network.
Notably,~\citet{learn_reweight} tackles noisy data filtering for image classification, by using meta-learning to calculate a locally optimized weight vector for each batch of data.
In contrast, our work focuses on the general problem of optimizing data usage. We train a parameterized scorer network that optimizes over the entire data space, which can be essential in preventing overfitting mentioned in Sec.~\ref{sec:method};  empirically our method outperform \cite{learn_reweight} by a large margin in Sec.~\ref{sec:experiment}.
%\gn{Here we should discuss \cite{learn_reweight} in detail, so that we can explain the difference for people who are familiar with this paper. Specifically (1) our work is parameterized and parameters are amortized over the entire dataset, which may be essential in preventing the overfitting mentioned in Section \ref{diff_data_selection}, (2) our work is not only focused at filtering noisy data, and (3) it empirically performs better.}
\citep{reinforce_cotrain,rl_nmt,learn_active_learn} propose RL frameworks for specific natural language processing tasks, but their methods are less generalizable and requires more complicated featurization.   
%The idea of selecting training data that are similar to dev data has been used in works on submodular optimization~\citep{submodular_mt,learn_mix_submodular}, but they rely on features specific to the task, while \dds~directly optimizes the the dev set performance, and is generalizable across tasks. Moreover, unlike \dds, these methods cannot adaptively change the data selection scheme. \cite{importance_weight} optimizes data weights by minimizing the error rate on the dev set. However, they use a single number to weigh each subgroup of augmented data, and their algorithm requires an expensive heuristic method to update data weights; while \dds~uses a more expressive parameterized neural network to model the individual data weights, which are efficiently updated by directly differentiating the dev set loss.




%Instance weighting: \cite{jiang-zhai-2007-instance,foster-etal-2010-discriminative,wang-etal-2017-instance}. In particular, \cite{wang-etal-2017-instance} seems like a good paper to compare against because it's recent and based on neural MT.

%Curriculum learning: \cite{zhang2016boosting,zhang2018empirical,platanios19naacl}.

%Data selection: \cite{moore2010intelligent,axelrod2011domain}.

%Removing bad training examples improves MT: \cite{vyas-etal-2018-identifying,pham-etal-2018-fixing}

%Dataset poisoning (for neural models): \cite{koh2017understanding}



%Meta-learning. Formulation of using the gradient update equation is similar to MAML \cite{finn2017model}.
