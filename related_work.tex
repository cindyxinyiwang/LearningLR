\section{\label{sec:related_work}Related Works}
%\gn{These are mostly based on stuff for NMT, a broader survey is necessary to increase more general ML methods. I've also only added a small subset of the papers on MT. Please follow the backward and forward references (forward references can be found by clicking the ``cited by'' link in Google Scholar).}

Selecting good training data is an important research topic in both Natural Language Processing and Computer Vision. For NMT, several prior work has focused on data selection for domain adaptation~\citep{moore2010intelligent,axelrod2011domain}, generally using heuristics to measure domain similarity and select the data above a predefined similarity threshold. Others have proposed to use all the data but weigh each instance according to heuristically defined domain similarity score~\citep{jiang-zhai-2007-instance,foster-etal-2010-discriminative,wang-etal-2017-instance}. Besides domain adaptation, it is also found that selecting good examples from the training data can improve NMT~\citep{vyas-etal-2018-identifying,pham-etal-2018-fixing}. All of the above methods involve certain heuristic designs using domain knowledge, while DDS can directly optimize the data selection strategy without heuristic estimation. Recently, \cite{domain_adapt_transfer} propose to estimate the importance weight of the classification labels in the pretraining dataset to mitigate the domain differences between pretraining and fine-tuning, while DDS is a more general data selection framework that works for both classification and other usage cases.  

The formulation of DDS involves bilevel optimization~\citep{bilevel_optim,hier_optim}, which is utilized in several prior work in areas other than data selection~\citep{darts,hyper_grad,finn2017model}. More generally, our method is also related to curriculum learning. Human designed training curriculum is found to be very effective for NMT~\citep{zhang2016boosting,zhang2018empirical,platanios19naacl}. \cite{baysian_curriculum} proposed to use Bayesian Optimization to learn the optimal curriculum. However, it simply learns the optimal combination of predefined heuristic features, while our method does not need any domain-specific knowledge and thus can be generalized to any deep learning models.    

Lastly, our method is related to the work on investigating the effect of label noise on deep image recognition models \cite{rolnick2017deep,overfit_random_examples} or machine translation \cite{khayrallah-koehn-2018-impact}, while \citet{koh2017understanding} shows the feasibility of training set poisoning attacks.


%Instance weighting: \cite{jiang-zhai-2007-instance,foster-etal-2010-discriminative,wang-etal-2017-instance}. In particular, \cite{wang-etal-2017-instance} seems like a good paper to compare against because it's recent and based on neural MT.

%Curriculum learning: \cite{zhang2016boosting,zhang2018empirical,platanios19naacl}.

%Data selection: \cite{moore2010intelligent,axelrod2011domain}.

%Removing bad training examples improves MT: \cite{vyas-etal-2018-identifying,pham-etal-2018-fixing}

%Dataset poisoning (for neural models): \cite{koh2017understanding}



%Meta-learning. Formulation of using the gradient update equation is similar to MAML \cite{finn2017model}.